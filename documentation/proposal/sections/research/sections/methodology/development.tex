
{\let\clearpage\relax \chapter{Development Methodology}}

Even though this project has few clearly defined requirements, designing and developing them will require an iterative model as there isn't a single best way to develop this and the author will be experimenting with different techniques. Thus the author decides on using \textbf{prototyping} as the \ac{sdlc} Model for this project.\\

\section{Design Methodology}

To design the system diagrams for this project \ac{ooad} methods will be used. \ac{ooad} make it easier to design the system iterative and this complement the choice \ac{sdlc} method Prototyping.

\section{Evaluation Methodology}

During the literature, the survey author concluded that there are not any specific evaluation metrics for the root cause analysis system other than accuracy and f1 score, and there are not any publicly available datasets or systems to benchmark against. Base-level benchmarks will be carried out to compare the proposed system with the existing ones. 

\section{Requirements Elicitation}

As the results of this project will be mostly used by \acp{sres} and system administrator the author is hoping to talk with few of the experts in the respective fields to get a better idea on what are the things to be expected from a system like this. Moreover as mentioned in \ref{sec:out-scope} this system is not designed to entirely replace existing monitoring systems, So the author is hoping to research about production monitoring systems and their workflows to understand how the proposed system could seamlessly integrate them. 
