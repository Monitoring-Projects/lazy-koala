

{\let\clearpage\relax \chapter{Problem Background}}
% {\let\clearpage\relax \chapter{Problem Domain}}

\section{Cloud Computing}
With an emergence \ac{iaas} like Amazon Web Services (AWS) and Google Cloud Platform (GCP) there is a big surge in organizations trying to outsource their computing needs to third parties \citep{rimol_2021}. This is mainly due to the elasticity given by all the cloud providers. Users can easily scale up and down their infrastructures within minutes without making any commitment and all the major providers, bill users on what you use are what you pay model because cloud provider manages all the underlying infrastructure users doesn't have to worry about problems like hardware failures. In contrast in a self-hosted setting if the user wanted one extra GB of memory than what's available it requires a lot of effort and cost to full fill that requirement.

\section{Cloud-Native Applications}
During the 90s and early 2000s, all the applications were made as a big monolith from a single code base \citep{LessonsF52:online}. Most of them were shipped as a single binary. Since those days applications were fairly simple this worked very well with little to no downsides. But when the 2010s came around there were a lot of specialized frameworks and programming languages and marketing teams wanted a lot of new futures quickly developed still maintaining reliability \citep{di2018migrating,Microser52:online}. But if the code base of the application was stored in a single repository, developers have to go through a long process to review and test if the changes won't break the current system and developers are also limited by the framework and programming language initial develops chosen for the project.

To tackle these problems there was a new way to develop applications was introduced, it's called "Microservices". The idea behind this concept is to break all the functionalities of big monolith applications into small individually scalable services and give ownership of each service to small teams of people who work separately. With this flow developers are free to use whatever tool they like to develop each service. Because these services are developed parallelly by different teams this increases the development velocity by order of magnitude \citep{Understa56:online}.

As these services are relatively small and tailor-made to run on cloud environments it's very easy to take something that's running on the developer's local machine to the production cluster in a matter of minutes. This is mainly thanks to modern cloud-native tools like CI/CD pipelines which automatically build and test the code for them, which can save a lot of time spent just doing repetitive tasks which are prone to human errors \citep{Whataret68:online}.

\section{Monitoring Cloud-Native Applications} \label{monitoring-bg}
Even though cloud-native applications have a lot to offer when it comes to developer velocity and productivity, It has its fair share of issues. Most of these problems are linked to the sheer complexity of these systems and not having a proper way to monitor them \citep{5WaysYou35:online}. All 3 major cloud providers provide a way to monitor these applications efficiently and some great open-source projects do this well, But to take full advantage of those systems, developers have to adapt their services to export all the vitals in a way the monitoring system understand. This works for the most part and this is what all the big companies are doing, even if it takes more developer time to in the end it's very crucial when it comes to disaster recovery.

But there is still a slight problem with this approach. Once the system starts to scale up to 100s of services number vitals that has to be monitored goes to 1000s and will require a lot of additional \acp{sres} and will have drop lot of non-crucial service vitals and derive abstract \acp{sli} to make it \textbf{humanly} possible to understand what's going on.\\
