\chapter*{Abstract}

Cloud computing has shown considerable growth due to its scalability and convenience in the past few years. With this change, a new programming paradigm called cloud-native originated. Cloud-native applications are often developed as a set of stand-alone microservices yet could depend on each other to provide a unified experience. Although microservices introduce many benefits regarding flexibility and scalability, it could be a great affliction to operate in production. Specifically, when operating a large system with hundreds of microservices interacting with each other, even the slightest problem could result in failures throughout the system.

The foci of this project are twofold. First, the authors introduce a robust Kubernetes native toolkit that helps researchers and developers collect and process service telemetry data with zero user instrumentation. Secondly, the authors proposed a novel way to detect anomalies by encoding raw metric data into an image-like structure and using a convolutional autoencoder to acquire the knowledge of the general data distribution for each service and detect outliers. Finally, a directed graph was used along with anomaly scores calculated before to show the spread of an anomaly to the user visually. 

After an extensive testing and evaluation process, it was found that the telemetry extraction components are both resilient and lightweight even under sustained load. At the same time, the anomaly prediction algorithm is accurate and generalizable.
\newline
\newline
\textbf{Keywords}:
AIOps, Monitoring, Disaster Recovery, eBPF, Kubernetes
\newline
\textbf{Subject Descriptors}:
• Computing methodologies $\rightarrow$ Machine learning $\rightarrow$ Learning paradigms $\rightarrow$ Unsupervised learning $\rightarrow$ Anomaly detection • Computer systems organization $\rightarrow$ Architectures $\rightarrow$ Distributed architectures $\rightarrow$ Cloud computing